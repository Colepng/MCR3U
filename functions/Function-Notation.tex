\documentclass{article}
\usepackage{amsmath}
\usepackage{graphicx}
\usepackage{hyperref}
\hypersetup{colorlinks=true, linkcolor=black}
\usepackage[latin1]{inputenc}
% \usepackage{cancel}
\usepackage{multicol}
\usepackage{pgfplots}
\usepackage{indentfirst}
\usepackage{enumitem}
\pgfplotsset{axis lines=left, grid style={solid,black},grid=both,xtick distance=2, ytick distance =2, xmin=-10,xmax=10,ymin=-10,ymax=10}%xtick={-10,-9,-8,-7,-6,-5,-4,-3,-2,-1,0,1,2,3,4,5,6,7,8,9,10},ytick={-10,-9,-8,-7,-6,-5,-4,-3,-2,-1,0,1,2,3,4,5,6,7,8,9,10}}
\pgfplotsset{compat=1.18}

\begin{document}
\title{Function-Notation}
% \author{Cole Kauder-McMurrich}
% \date{\today}
% \maketitle

\section*{Find Values Using Function Notation}

\subsection*{What does a function do?}
Takes an input(x), performs operations on it and then gives an output (y)

\subsection*{What does function notation look like?}
$f(x)$ = ... something to do with x

read as $f$ at $x$ or $f$ of $x$

replaces $y$

\subsection*{Example 1}
For each of the following functions, determine $f(2)$,$f(-5)$, and $f(1/2)$\\
% \begin{enumerate}[label=\alph*)]
    % \item $f(x) = 2x - 4$
\indent
a) $f(x) = 2x - 4$

\begin{multicols}{3}
\noindent
\begin{equation*}
\begin{split}
    f(2) &= 2(2) - 4\\
    &= 4 - 4\\
    &= 0\\
\end{split}
\end{equation*}
\begin{equation*}
\begin{split}
    f(5) &= 2(5) - 4\\
    &= 10 - 4\\
    &= 6\\
\end{split}
\end{equation*}
\begin{equation*}
\begin{split}
    f\left(\frac{1}{2}\right) &= 2\left(\frac{1}{2}\right) - 4\\
    &= 1 - 4\\
    &= -3\\
\end{split}
\end{equation*}
\end{multicols}

b) $f(x) = 3x^2 - x + 7$

\begin{multicols}{2}
\noindent
\begin{equation*}
\begin{split}
    f(2) &= 3(2)^2 - 2 + 7\\
    &= 12 - 2 + 7\\
    &= 10 + 7\\
    &= 17\\
\end{split}
\end{equation*}
\begin{equation*}
\begin{split}
    f(5) &= 3(-5)^2 - (-5) + 7\\
    &= 75 - (-5) + 7\\
    &= 80 + 7\\
    &= 87\\
\end{split}
\end{equation*}
\end{multicols}
\begin{equation*}
\begin{split}
    f\left(\frac{1}{2}\right) &= 3\left(\frac{1}{2}\right)^2 - \left(\frac{1}{2}\right) + 7\\
    &= \frac{3}{4} - \left(\frac{1}{2}\right) + 7\\
    &= \frac{1}{4} + 7\\
    &= 7\frac{1}{4}\\
\end{split}
\end{equation*}

\newpage
\section*{Applications of Function Notation}
For the function $h(t) = -3(t+1)^2 + 5$\\
\indent
a) Graph it and find the domain and range\\
\indent
skipping because graphing is being annoying
% \begin{tikzpicture}
%     \begin{axis}
%         \addplot[blue, ultra thick] {-3(x+1)^2+5};
%         % \addplot[color=blue] coordinates{(-14,-10)(10, 14)};
%         % \addplot[color=red] coordinates{(12,-10)(-10, 12)};
%         % \addlegendentry{\(y=x + 4\)}
%         % \addlegendentry{\(y=-x + 2\)}
%     \end{axis}
% \end{tikzpicture}
b) Find $h(-7)$

\begin{equation*}
\begin{split}
    h(-7) &= -3(-7+1)^2+5\\
          &= -3(36)+5\\
          &= -108+5\\
          &= -103\\
\end{split}
\end{equation*}
\end{document}
